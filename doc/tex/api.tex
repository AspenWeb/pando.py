\chapter{Application Programming Interface (API) \label{api}}

Aspen parses and harmonizes all command-line, configuration file, and
environment settings before it loads your plugins. All of this information is
then available to your modules via several objects which are dynamically placed
in the global \module{aspen} namespace before your plugins are loaded.


\section{The \class{aspen.configuration} object}
\label{api-configuration}

The \code{aspen.configuration} object provides raw access to the parser objects
used to configure your server, and a number of basic settings.

\subsection{Parsers}

The various parsers and raw settings are exposed as these members:

\begin{memberdesc}[list]{args}
An argument list as returned by \code{optparse.OptionParser.parse_args}.
\end{memberdesc}

\begin{memberdesc}[ConfFile]{conf}
An instance of \code{aspen._configuration.ConfFile}; see below.
\end{memberdesc}

\begin{memberdesc}[OptionParser]{optparser}
An \code{optparse.OptionParser} instance.
\end{memberdesc}

\begin{memberdesc}[Values]{opts}
An \code{optparse.Values} instance per \code{optparse.OptionParser.parse_args}.
\end{memberdesc}

\begin{memberdesc}[Paths]{paths}
An instance of \code{aspen._configuration.Paths}; see below.
\end{memberdesc}


\subsection{Settings}

Furthermore, \code{aspen.configuration} exposes specific configuration settings
as these members:

\begin{memberdesc}[]{address}
A (\var{hostname}, \var{port}) tuple (for \code{AF_INET} and \code{AF_INET6}
address) or string (for \code{AF_UNIX}) giving the address to which Aspen is
bound.
\end{memberdesc}

\begin{memberdesc}[string]{command}
A string giving the command line argument (\var{start}, \var{stop}, etc.).
\end{memberdesc}

\begin{memberdesc}[boolean]{daemon}
A boolean indicating whether Aspen is acting as a daemon.
\end{memberdesc}

\begin{memberdesc}[tuple]{defaults}
A tuple listing the default resource names to look for in a directory.
\end{memberdesc}

\begin{memberdesc}[int]{sockfam}
One of \code{socket.AF_INET}, \code{socket.AF_INET6}, and \code{socket.AF_UNIX}.
\end{memberdesc}

\begin{memberdesc}[int]{threads}
A non-zero positive integer; the number of threads in the server's
request-handling thread pool.
\end{memberdesc}


All members are intended to be read-only.


\begin{seealso}

\seelink{http://docs.python.org/lib/module-ConfigParser.html}
{\code{ConfigParser}}{The naming is not PEP 8, but the documentation is fine.}

\seelink{http://docs.python.org/lib/module-optparse.html} {\code{optparse}}{On
the other hand, the documentation for \module{optparse} is rather, um,
convoluted. Good luck!}

\end{seealso}


\section{The \class{aspen.conf} object}
\label{api-conf}

The \class{aspen.conf} object is an instance of
\class{aspen._configuration.ConfFile}, which subclasses the standard library's
\class{ConfigParser.RawConfigParser} class to represent the
\file{__/etc/aspen.conf} file. In addition to the \class{RawConfigParser} API,
the object supports both attribute and key read-only access; either returns a
dictionary corresponding to a section of the \file{aspen.conf} file. If the
named section does not exist, an empty dictionary is returned.

Your application is free and encouraged to use the \file{aspen.conf} file for
it's own configuration, and to access that information via this object.

To illustrate, here is a minimal \file{aspen.conf} file:

\begin{verbatim}
[my_settings]
foo = bar
\end{verbatim}

Such a file could support code like this:

\begin{verbatim}
import aspen

def wsgi_app(environ, start_response):
    my_setting = aspen.conf.my_settings.get('foo', 'default')
    start_response('200 OK', [])
    return ["My setting is %s" % my_setting]
\end{verbatim}


\begin{seealso}

\seelink{http://docs.python.org/lib/RawConfigParser-objects.html}
{\code{RawConfigParser}}{In addition to the API above, \code{aspen.conf} also
exposes the \code{RawConfigParser} API.}

\end{seealso}



\section{The \class{aspen.paths} object}
\label{api-paths}

The \class{aspen.paths} object is an instance of
\class{aspen._configuration.Paths}; it is simply a container for various paths,
all normalized and absolute:

\begin{memberdesc}[string]{root}
the website's filesystem root
\end{memberdesc}

\begin{memberdesc}[string]{__}
the magic directory
\end{memberdesc}

\begin{memberdesc}[string]{lib}
the site's local Python library
\end{memberdesc}

\begin{memberdesc}[string]{plat}
the local platform-specific Python library
\end{memberdesc}
