\section{Middleware}

The Aspen package bundles the following WSGI middleware in the
\module{aspen.middleware} subpackage.


\subsection{\module{raised}}

\declaremodule{aspen}{raised}

The \module{aspen.middleware.raised} module provides for ending WSGI requests by
raising a \class{Response} object that is caught by a middleware:

\begin{classdesc}{Response}{\optional{code} \optional{, body} \optional{,
    headers}}
Constructs a new \class{Response} object. If given, \var{code} must be an
integer; the default is
\ulink{200}{http://www.w3.org/Protocols/rfc2616/rfc2616-sec10.html#sec10.2.1}
(see \ulink{the HTTP
spec}{http://www.w3.org/Protocols/rfc2616/rfc2616-sec10.html} for other values
that will be meaningful to most HTTP clients). \var{headers} may be a dictionary
or a list of 2-tuples. \var{body} may be a string or other iterable.
\end{classdesc}

\begin{funcdesc}{middleware}{next}
WSGI middleware; \var{next} is the next WSGI callable on the stack. This
middleware catches any \class{Response} objects raised by \var{next} and calls
them (they are themselves WSGI callables) to produce the response.
\end{funcdesc}


\subsubsection{\class{Response} Objects}

Instances of \class{aspen.middleware.raised.Response} are WSGI applications,
with the following data attributes. Note that values are only validated in the
constructor, so it is possible to raise a malformed \class{Response} by setting
instance attributes post-instantiation.

\begin{datadesc}{code}
The \ulink{HTTP code}{http://www.w3.org/Protocols/rfc2616/rfc2616-sec10.html} as an integer.
\end{datadesc}

\begin{datadesc}{body}
The message body as a string or other iterable.
\end{datadesc}

\begin{datadesc}{headers}
The message headers as an instance of the standard library's
\ulink{\module{email.Message.Message}}{http://docs.python.org/lib/module-email.message.html}.
\end{datadesc}

When called, \class{Response} instances call \code{start_response} with
adaptations of \var{code} and \var{headers}, and return \var{body}, or a 
one-item list containing \var{body} if \var{body} is a string.


\subsubsection{Example}

Here is an example:

\begin{verbatim}
Python 2.5 (r25:52005, Sep 25 2006, 21:37:36)
[GCC 3.4.4 [FreeBSD] 20050518] on freebsd6
Type "help", "copyright", "credits" or "license" for more information.
>>>
>>> from aspen import raised
>>> def app(env, start):
>>>     raise raised.Response(200, "Greetings, program!")
>>> app = raised.middleware(app)
>>>
>>> from wsgiref.simple_server import make_server
>>> server = make_server('', 8080, app)
>>> server.serve_forever() # now hit http://localhost:8080/
>>>
192.168.1.100 - - [09/Nov/2006 23:52:45] "GET / HTTP/1.1" 200 19
\end{verbatim}
